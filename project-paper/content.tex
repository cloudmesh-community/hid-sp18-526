% status: 100
% chapter: REST, Containers

\title{CMENV: Deployable Cloudmesh Containers}

\author{Tim Whitson}
\affiliation{
  \institution{Indiana University}
  \streetaddress{Smith Research Center}
  \city{Bloomington}
  \state{IN}
  \postcode{47408}
  \country{USA}}
\email{tdwhitso@indiana.edu}

\author{Gregor von Laszewski}
\affiliation{
  \institution{Indiana University}
  \streetaddress{Smith Research Center}
  \city{Bloomington}
  \state{IN}
  \postcode{47408}
  \country{USA}}
\email{laszewski@gmail.com}

% The default list of authors is too long for headers}
\renewcommand{\shortauthors}{T. Whitson}

% program name (subject to change)
\def\projectname{\textit{cmenv}}

\begin{abstract}
In order to meet the needs of modern cloud computing environments, there is a
need for flexible, language-independent computing resources. To meet this
need, we present \projectname, a modularized system of REST APIs housed
within easily-deployable Docker containers. Each individual API stands alone
within the program and we discuss the structure for APIs. A custom resolver
is created to route HTTP requests back to the API controllers. Various REST
services are given as standard for maintaing the program.
\end{abstract}

\keywords{hid-sp18-526, cloudmesh, rest, swagger}

\maketitle

\section{Introduction}

Docker is a program that provides containerization (virtualization
within the operating system). Docker is quickly becoming standard for
software development and deployment within the cloud. According to the
2016 Docker Survey, 60\% of users deploy Docker as part of their ``cloud
strategy''\cite{hid-sp18-526-www-docker-survey}.

REST APIs are another crucial cloud component. In fact, many important software
programs, such as Kubernetes, come with REST APIs built-in. An important
benefit of REST APIs is that they are language-independent. All web languages
have methods for interacting with, and creating, REST APIs. Therefore, REST
APIs play a crucial role in \projectname.

We combine all required services into a single REST API that can be deployed
easily within a Docker container. Each API stands on its own as a modular
system.

\section{OpenAPI}

It is important, especially when working with modular systems, that all
components of the system adhere to the same standard. For this project,
we use OpenAPI. According to their website, OpenAPI ``defines a standard,
programming language-agnostic interface description for REST APIs, which allows
both humans and computers to discover and understand the capabilities of a
service without requiring access to source code, additional documentation,
or inspection of network traffic.''\cite{hid-sp18-526-www-openapi}

The specific OpenAPI standard we use is Swagger 2.0. Swagger 2.0 allows for
simple resource definitions. We will not require any operation handling,
as all routing will be done by a custom resolver, discussed later.

\section{Configuration}

Each instance of \projectname~uses its own configuration file,
\textit{config.yml}. This is due to the fact that the containers need to
be composable. Currently, the configuration allows for the selective use
of services. Either a list of services or ``all'' can be provided in the
configuration to determine which services are running. The container will
then run only the APIs which are listed. Currently, the setup also requires
the \textit{Dockerfile} to be in the install directory.

In order to launch the program, the following directory structure must be
adhered to:
\begin{verbatim}
.
|-- config.yml # CMENV configuration
|-- Dockerfile # Dockerfile

\end{verbatim}
\begin{itemize}
    \item \textbf{config.yml} This file contains the configuration for
    \projectname, such as which services will be run.
    \item \textbf{Dockerfile} This Dockerfile is provided. It should not be
    modified except in special circumstances.
\end{itemize}

\section{Program Functionality}

The first role of the program is to generate the Swagger specification. In
order to do this, all requested services must be gathered. Then the
program combines the paths of each module from its \textit{swagger.yml}
specification. After all services are combined, a final Swagger specification
is created with required fields (title, version, etc.). This document is fed
into \textit{connexion}. The combined Swagger document can also be output
as either yaml or json.

\subsection{connexion}

connexion is a Python program written by Zalando SE to combine OpenAPI and
Flask~\cite{hid-sp18-526-www-connexion}. Effectively, connexion takes the
place of Swagger Codegen. It accepts a Swagger specification and creates
a Flask server with pre-generated routes. Typically, with connexion, the
user would have to specify the controllers manually or allow the connexion
\textit{resty resolver} to automatically route the requests. However, due
to the modularity of \projectname, we will create our own router. See the
Routing section for more information.

\section{API Structure}

Because \projectname~is a composable system, each part of the API should
be independent. Therefore, we use a modular system of self-contained APIs,
which are then combined into a single server. This allows for a ``plugin''
(and plug out) system where new API modules can easily be created and added
to the program.

Directory layout:
\begin{verbatim}
.
|-- swagger.yml # Swagger specification
|-- requirements.txt # python requirements
|-- packages.txt # Ubuntu package requirements
|-- controllers # dpython controllers for API
    |-- ...

\end{verbatim}

\begin{itemize}
    \item\textbf{swagger.yml} This file contains the Swagger specification,
    for this API only (paths, definitions, etc.). The top-level Swagger
    specification (Swagger version, API title) are handled by the main program.

    \item\textbf{requirements.txt} This file is a newline-separated list of
    python package requirements, to be installed via \textit{pip3 install}
    (currently all packages use Python 3).

    \item\textbf{pakackages.txt} This is a newline-separated list of Ubuntu
    package requirements, to be installed via \textit{apt-get install}.

    \item\textbf{controllers} This directory is where the API controllers are
    placed, with name corresponding to the path. Each file in this directory
    is a Python file which acts as a controller, defined by the path names
    from swagger.yml. See Routing section for more information.
\end{itemize}

Another reason for following such a structure is to properly manage
dependencies. Each API has its own dependencies, both in the operating system
and in the programming language (Python). With a modular setup, individual
dependencies can be assigned and handled by the program. Therefore, no master
list of dependencies is required, and individual modules can be installed
or uninstalled along with their dependencies.

\section{Routing}

To fit with the modularized system of APIs, a custom router, or resolver,
is used. The router needs to be able to find each individual API, and call
the controllers located within the API. First, a one-to-one mapping of
modules to paths is created, to ensure that the controller can refer to the
original module. Then, the method is placed as the endpoint. In keeping with
the connexion package standard, a GET request to the base path is referred
to as ``search''.

For example, if in the \textit{store} module, we have the
path ``/key'' and method ``GET'', the request is routed to
apis.store.controllers.key.search. This leads to the \textit{store} module, the
``controller'' subdirectory, the ``key.py'' file, and the ``search'' function.

Example using \textit{store} module:
\begin{verbatim}
paths:
  /key:
    get:
      # route: apis.store.controllers.store.search

  '/key/{key}':
    get:
      # route: apis.store.controllers.store.get
    delete:
      # route: apis.store.controllers.store.delete

  '/key/{key}/{value}':
    put:
      # route: apis.store.controllers.store.put
\end{verbatim}

\section{APIs}

The purpose of \projectname~is to provide an arbitrary number of
services. However, a few services come packaged. These services, which are
outlined here, are crucial to maintaining the operating environment.

\subsection{Key-Value Store}

A Key-value store is implemented, using TinyDB. TinyDB is a simple, lightweight
local file store written in Python. Essentially, it acts like a local version
of MongoDB. Each item is stored as a json document, and all documents for a
``db'' are combined into a json file. These files are read in and out of
Python and easily dumped or loaded as dictionaries. This solution provides
an efficient, queryable file store with no dependencies on the operating
system.\cite{hid-sp18-526-www-tinydb}

\subsection{Services}

A services library is necessary to manage the API modules. The library handles
not only running services, but the management of services as well. This
will be useful for changing the initial configuration parameters after the
container has been created. The API can talk to the program and turn off
and on whichever services are requested (requiring a restart).

\section{Future Improvements}

This project is a work-in-progress. Many more changes and adjustments will
need to be made for full implementation. In this discussion we discuss the
future of the project.

\subsection{Core APIs}

More core APIs need to be added for maintenance and functionality. The
following core APIs will be added in the future:
\begin{itemize}
    \item \textbf{key management} This API will manage SSH/public keys.
    \item \textbf{accounting} This API will enable accounting of services
    and resources. Also included will be a logging system.
    \item \textbf{data services} This API will manage a file and object
    store. There will also be virtual files and virtual objects. This service
    might also benefit from TinyDB.
\end{itemize}

\subsection{Service Monitor}

A service monitor would be an important asset for the combined APIs. Such a
monitor could run in a browser and have the capability to start/stop services
and generate URLs or parameters. An HTML template could be created using
Jinja and Flask to communicate with any running API.

\subsection{Remote Setup}

A remote setup option is possible, without requiring a local directory,
Dockerfile, or configuration file. However, This setup would be a more
standardized setup, with all services running. The user would be required
to manually configure the setup through the \textit{services} module.

\section{Conclusion}

\projectname~is a modern solution to a modern problem. We are able to deploy
a container of modularized and combined REST APIs easily and efficiently. We
use connexion and OpenAPI specifications to create a combined specification of
all services running on a Flask server. We create a custom router to route HTTP
requests to the modules. There is still more work to do on \projectname, which
has been outlined in the paper. However, we believe that \projectname~has
a solid foundation moving forward to simplifying distributed and cloud
computing problems.

\bibliographystyle{ACM-Reference-Format}
\bibliography{report}

