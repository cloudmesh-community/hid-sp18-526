% status: 5
% chapter: TBD

\title{CMENV: Deployable Cloudmesh Containers}

\author{Tim Whitson}
\affiliation{%
  \institution{Indiana University}
  \streetaddress{Smith Research Center}
  \city{Bloomington}
  \state{IN}
  \postcode{47408}
  \country{USA}}
\email{tdwhitso@indiana.edu}

\author{Gregor von Laszewski}
\affiliation{%
  \institution{Indiana University}
  \streetaddress{Smith Research Center}
  \city{Bloomington}
  \state{IN}
  \postcode{47408}
  \country{USA}}
\email{laszewski@gmail.com}

% The default list of authors is too long for headers}
\renewcommand{\shortauthors}{T. Whitson}

% program name (subject to change)
\def \projectname {\textit{cmenv}}

\begin{abstract}
In order meet the needs of modern cloud computing environments, there is a
need for flexible, language-independent computing resources. To meet this
need, we present \projectname, a modularized system of REST APIs housed
within easily deployable Docker containers.
\end{abstract}

\keywords{hid-sp18-526, cloudmesh, rest, swagger}

\maketitle

\section{Introduction}

Docker is a program that provides containerization (virtualization
within the operating system). Docker is quickly becoming standard for
software development and deployment within the cloud. According to the
2016 Docker Survey, 60\% of users deploy Docker as part of their ``cloud
strategy''\cite{hid-sp18-526-docker-survey}.

REST APIs are another crucial cloud component. In fact, many important software
programs, such as Kubernetes, come with REST APIs built-in. An important
benefit of REST APIs is that they are language-independent. All web languages
have methods for interacting with, and creating, REST APIs. Therefore, REST
APIs play a crucial role in \projectname. We will combine all required services
into a single REST API that can be deployed easily within a Docker container.

\section{OpenAPI}

It is important, especially when working with modular systems, that all
components of the system adhere to the same standard. For this project,
we use OpenAPI. According to their website, OpenAPI ``defines a standard,
programming language-agnostic interface description for REST APIs, which allows
both humans and computers to discover and understand the capabilities of a
service without requiring access to source code, additional documentation,
or inspection of network traffic.''\cite{hid-sp18-526-openapi}

\section{Configuration}

\section{Public Keys}

\section{Key-Value Store}

\section{Data Services}

% cite

\bibliographystyle{ACM-Reference-Format}
\bibliography{report}

